\providecommand{\main}{.}
\input{\main/../../general/packages.tex}
\usepackage{booktabs}
\begin{document}

\input{\main/../../general/title.tex}

{\hypersetup{hidelinks}
	\tableofcontents  % Generates the table of contents
}
\part{Dataset}
\chapter{Data points}
First we begin looking at the dataset, the distributions of the given metrics and the statistical analysis of these data points.

\section{Retrieving the dataset}
The dataset can be downloaded from \url{https://homes.di.unimi.it/valentini/ProgettoBioinformatica1718/data/}.

\section{Composition}
\subsection{Training dataset}
In the training dataset there are 981389 data points, each one comprised of 26 metrics. The first 356 are pathogenic and all the others are negative.

\subsection{Testing dataset}
In the test dataset there are 190189 data points, still each one comprised of 26 metrics. The first 40 are pathogenic and the following are negative.

\chapter{Metrics}
\section{How the graphs are realized}
All the graphs are in triples: positives, negatives and mixed. All the zeros are removed as in most metrics \textit{seemed} to indicate an unknown value.

\subsection{Metric sample distribution}
Are realized by calculating the frequencies and estimating the density distributions parameters via MLE.

\subsection{Plot graphs}
Are realized by sorting the values of the metric.

\subsection{Normalized plot graphs}
Are realized by sorting the values of the metric, with the domain and codomain normalized.

\clearpage
\section{CpGobsExp}
\subsection{Metric sample distribution}
The data points seem to follow a \textbf{Beta} distribution with the following parameters:

\begin{align*}
	\alpha   = 7.6689746880295795     & \qquad  \beta = 6778383.524935903       \\
	\text{loc} = -0.09826818916997124 & \qquad \text{scale} = 306278.3184506849
\end{align*}

\begin{figure}
	\includegraphics[width=\textwidth]{metrics_statistics/CpGobsExp}
	\caption{Sampling distribution of metric CpGobsExp}
\end{figure}
\subsection{Metric values}
\begin{figure}
	\includegraphics[width=\textwidth]{metrics_plot/CpGobsExp}
	\caption{Values of metric CpGobsExp}
\end{figure}

\clearpage
\section{CpGperCpG}
\subsection{Metric sample distribution}
The data points seem to follow a \textbf{Beta} distribution with the following parameters:

\begin{align*}
	\alpha   = 6.402175341881067      & \qquad  \beta = 97129163.31117742      \\
	\text{loc} = -0.05698922703576313 & \qquad \text{scale} = 4337764.42876015
\end{align*}
\begin{figure}
	\includegraphics[width=\textwidth]{metrics_statistics/CpGperCpG}
	\caption{Sampling distribution of metric CpGperCpG}
\end{figure}
\subsection{Metric values}
\begin{figure}
	\includegraphics[width=\textwidth]{metrics_plot/CpGperCpG}
	\caption{Values of metric CpGperCpG}
\end{figure}

\clearpage
\section{CpGperGC}
\subsection{Metric sample distribution}
The data points seem to follow a \textbf{Gaussian} distribution with the following parameters:

\begin{align*}
	\mean{X} = 0.4602356242601636 & \qquad \Var{X} = 0.15949294643574352
\end{align*}
\begin{figure}
	\includegraphics[width=\textwidth]{metrics_statistics/CpGperGC}
	\caption{Sampling distribution of metric CpGperGC}
\end{figure}
\subsection{Metric values}
\begin{figure}
	\includegraphics[width=\textwidth]{metrics_plot/CpGperGC}
	\caption{Values of metric CpGperGC}
\end{figure}

\clearpage
\section{DGVCount}
\subsection{Metric sample distribution}
The data points seem to follow a \textbf{Gamma} distribution with the following parameters:
\begin{align*}
	\alpha   = 0.20940038672579409    \qquad  \text{loc} = -1.1962983066939984e-30 \qquad \text{scale} = 1.2347090894162929
\end{align*}
\begin{figure}
	\includegraphics[width=\textwidth]{metrics_statistics/DGVCount}
	\caption{Sampling distribution of metric DGVCount}
\end{figure}
\subsection{Metric values}
\begin{figure}
	\includegraphics[width=\textwidth]{metrics_plot/DGVCount}
	\caption{Values of metric DGVCount}
\end{figure}

\clearpage
\section{DnaseClusteredHyp}
\subsection{Metric sample distribution}
The data points seem to follow a \textbf{Gamma} distribution with the following parameters:
\begin{align*}
	\alpha   = 0.4176887081406805    \qquad  \text{loc} = -3.362626207862299e-29 \qquad \text{scale} = 0.3676310948709975
\end{align*}
\begin{figure}
	\includegraphics[width=\textwidth]{metrics_statistics/DnaseClusteredHyp}
	\caption{Sampling distribution of metric DnaseClusteredHyp}
\end{figure}
\subsection{Metric values}
\begin{figure}
	\includegraphics[width=\textwidth]{metrics_plot/DnaseClusteredHyp}
	\caption{Values of metric DnaseClusteredHyp}
\end{figure}

\clearpage
\section{DnaseClusteredScore}
\subsection{Metric sample distribution}
The data points seem to follow \textbf{slightly} a \textbf{Beta} distribution with the following parameters:
\begin{align*}
	\alpha   = 0.2709657632937803     & \qquad  \beta = 0.44530002562349713      \\
	\text{loc} = -0.09309893086089688 & \qquad \text{scale} = 1.0930989308608972
\end{align*}

\begin{figure}
	\includegraphics[width=\textwidth]{metrics_statistics/DnaseClusteredScore}
	\caption{Sampling distribution of metric DnaseClusteredScore}
\end{figure}
\subsection{Metric values}
\begin{figure}
	\includegraphics[width=\textwidth]{metrics_plot/DnaseClusteredScore}
	\caption{Values of metric DnaseClusteredScore}
\end{figure}

\clearpage
\section{EncH3K27Ac}
\subsection{Metric sample distribution}
The data points seem to follow a family of \textbf{Gamma} distributions (a speculation for this distribution could be the different groups from which the data are extracted), we will approximate them to one with a linear combination of the parameters:
\begin{align*}
	\alpha   = 0.0004042086221537893    \qquad  \text{loc} = -2.859398162696207e-24 \qquad \text{scale} = 0.03076944787133299
\end{align*}
\begin{figure}
	\includegraphics[width=\textwidth]{metrics_statistics/EncH3K27Ac}
	\caption{Sampling distribution of metric EncH3K27Ac}
\end{figure}
\subsection{Metric values}
\begin{figure}
	\includegraphics[width=\textwidth]{metrics_plot/EncH3K27Ac}
	\caption{Values of metric EncH3K27Ac}
\end{figure}

\clearpage
\section{EncH3K4Me1}
\subsection{Metric sample distribution}
The data points seem to follow a family of \textbf{Gamma} distributions (a speculation for this distribution could be the different groups from which the data are extracted), we will approximate them to one with a linear combination of the parameters:
\begin{align*}
	\alpha   = 0.22566387737236238    \qquad  \text{loc} = -6.619765504581537e-27 \qquad \text{scale} = 1.396157055181753
\end{align*}
\begin{figure}
	\includegraphics[width=\textwidth]{metrics_statistics/EncH3K4Me1}
	\caption{Sampling distribution of metric EncH3K4Me1}
\end{figure}
\subsection{Metric values}
\begin{figure}
	\includegraphics[width=\textwidth]{metrics_plot/EncH3K4Me1}
	\caption{Values of metric EncH3K4Me1}
\end{figure}

\clearpage
\section{EncH3K4Me3}
\subsection{Metric sample distribution}
The data points seem to follow a family of \textbf{Gamma} distributions (a speculation for this distribution could be the different groups from which the data are extracted), we will approximate them to one with a linear combination of the parameters:
\begin{align*}
	\alpha   = 0.007502428717446465    \qquad  \text{loc} = -3.469650119186857e-25 \qquad \text{scale} = 0.04125297431971783
\end{align*}
\begin{figure}
	\includegraphics[width=\textwidth]{metrics_statistics/EncH3K4Me3}
	\caption{Sampling distribution of metric EncH3K4Me3}
\end{figure}
\subsection{Metric values}
\begin{figure}
	\includegraphics[width=\textwidth]{metrics_plot/EncH3K4Me3}
	\caption{Values of metric EncH3K4Me3}
\end{figure}

\clearpage
\section{GCContent}
\subsection{Metric sample distribution}
The data points seem to be a combination of two \textbf{Gaussian} distributions. This will be approximated to one with the following parameters:

\begin{align*}
	\mean{X} = 0.4482813176478024 & \qquad \Var{X} = 0.1097424869360011
\end{align*}
\begin{figure}
	\includegraphics[width=\textwidth]{metrics_statistics/GCContent}
	\caption{Sampling distribution of metric GCContent}
\end{figure}
\subsection{Metric values}
\begin{figure}
	\includegraphics[width=\textwidth]{metrics_plot/GCContent}
	\caption{Values of metric GCContent}
\end{figure}

\clearpage
\section{GerpRS}
\subsection{Metric sample distribution}
The data points seem to follow a family of \textbf{Gamma} distributions (a speculation for this distribution could be the different groups from which the data are extracted), we will approximate them to one with a linear combination of the parameters:
\begin{align*}
	\alpha   = 0.8688332877203315    \qquad  \text{loc} = -1.7081810436826354e-28 \qquad \text{scale} = 0.11512094125204281
\end{align*}
\begin{figure}
	\includegraphics[width=\textwidth]{metrics_statistics/GerpRS}
	\caption{Sampling distribution of metric GerpRS}
\end{figure}
\subsection{Metric values}
\begin{figure}
	\includegraphics[width=\textwidth]{metrics_plot/GerpRS}
	\caption{Values of metric GerpRS}
\end{figure}

\clearpage
\section{GerpRSpv}
\subsection{Metric sample distribution}
The data points seem to follow a family of \textbf{Gamma} distributions (a speculation for this distribution could be the different groups from which the data are extracted), we will approximate them to one with a linear combination of the parameters:
\begin{align*}
	\alpha   = 0.5165290213220888    \qquad  \text{loc} = -6.952792177974854e-30 \qquad \text{scale} = 0.2530358950266992
\end{align*}
\begin{figure}
	\includegraphics[width=\textwidth]{metrics_statistics/GerpRSpv}
	\caption{Sampling distribution of metric GerpRSpv}
\end{figure}
\subsection{Metric values}
\begin{figure}
	\includegraphics[width=\textwidth]{metrics_plot/GerpRSpv}
	\caption{Values of metric GerpRSpv}
\end{figure}

\clearpage
\section{ISCApath}
\subsection{Metric sample distribution}
The data points seem to follow a \textbf{Gamma} distribution with the following parameters:
\begin{align*}
	\alpha   = 0.08318618903703257    \qquad  \text{loc} = -1.9358902729364646e-30 \qquad \text{scale} = 1.2606790181148981
\end{align*}
\begin{figure}
	\includegraphics[width=\textwidth]{metrics_statistics/ISCApath}
	\caption{Sampling distribution of metric ISCApath}
\end{figure}
\subsection{Metric values}
\begin{figure}
	\includegraphics[width=\textwidth]{metrics_plot/ISCApath}
	\caption{Values of metric ISCApath}
\end{figure}

\clearpage
\section{commonVar}
\subsection{Metric sample distribution}
The data points seem to follow an \textbf{Exponential Weibull} distribution with the following parameters:

\begin{align*}
	\alpha   = 5.038707296051438       & \qquad  \beta = 1.0160276119461702         \\
	\text{loc} = -0.012528678364149837 & \qquad \text{scale} = 0.025052745155722922
\end{align*}
\begin{figure}
	\includegraphics[width=\textwidth]{metrics_statistics/commonVar}
	\caption{Sampling distribution of metric commonVar}
\end{figure}
\subsection{Metric values}
\begin{figure}
	\includegraphics[width=\textwidth]{metrics_plot/commonVar}
	\caption{Values of metric commonVar}
\end{figure}

\clearpage
\section{dbVARCount}
\subsection{Metric sample distribution}
The data points seem to follow a \textbf{Gamma} distribution with the following parameters:
\begin{align*}
	\alpha   = 0.20940038672579409    \qquad  \text{loc} = -1.1962983066939984e-30 \qquad \text{scale} = 1.2347090894162929
\end{align*}
\begin{figure}
	\includegraphics[width=\textwidth]{metrics_statistics/dbVARCount}
	\caption{Sampling distribution of metric dbVARCount}
\end{figure}
\subsection{Metric values}
\begin{figure}
	\includegraphics[width=\textwidth]{metrics_plot/dbVARCount}
	\caption{Values of metric dbVARCount}
\end{figure}

\clearpage
\section{fantom5Perm}
\subsection{Metric sample distribution}
The data points seem to follow a \textbf{Gamma} distribution with the following parameters:
\begin{align*}
	\alpha   = 0.06895533706017208    \qquad  \text{loc} = -3.220296247423778e-30 \qquad \text{scale} = 1.2605014923175824
\end{align*}
\begin{figure}
	\includegraphics[width=\textwidth]{metrics_statistics/fantom5Perm}
	\caption{Sampling distribution of metric fantom5Perm}
\end{figure}
\subsection{Metric values}
\begin{figure}
	\includegraphics[width=\textwidth]{metrics_plot/fantom5Perm}
	\caption{Values of metric fantom5Perm}
\end{figure}

\clearpage
\section{fantom5Robust}
\subsection{Metric sample distribution}
The data points seem to follow a \textbf{Gamma} distribution with the following parameters:
\begin{align*}
	\alpha   = 0.08983952110680529    \qquad  \text{loc} = -3.220296247423778e-30 \qquad \text{scale} = 1.2605014923175824
\end{align*}
\begin{figure}
	\includegraphics[width=\textwidth]{metrics_statistics/fantom5Robust}
	\caption{Sampling distribution of metric fantom5Robust}
\end{figure}
\subsection{Metric values}
\begin{figure}
	\includegraphics[width=\textwidth]{metrics_plot/fantom5Robust}
	\caption{Values of metric fantom5Robust}
\end{figure}

\clearpage
\section{fracRareCommon}
\subsection{Metric sample distribution}
The data points seem to follow an \textbf{Beta} distribution with the following parameters:

\begin{align*}
	\alpha   = 2772.739504773501    & \qquad  \beta = 14.986077009876375      \\
	\text{loc} = -69.93503912437342 & \qquad \text{scale} = 71.09741090721741
\end{align*}
\begin{figure}
	\includegraphics[width=\textwidth]{metrics_statistics/fracRareCommon}
	\caption{Sampling distribution of metric fracRareCommon}
\end{figure}
\subsection{Metric values}
\begin{figure}
	\includegraphics[width=\textwidth]{metrics_plot/fracRareCommon}
	\caption{Values of metric fracRareCommon}
\end{figure}

\clearpage
\section{mamPhastCons46way}
\subsection{Metric sample distribution}
The data points seem to follow a \textbf{Gamma} distribution with the following parameters:
\begin{align*}
	\alpha   = 0.3215099801387991    \qquad  \text{loc} = -6.260887365023215e-31 \qquad \text{scale} = 0.45230902834164866
\end{align*}
\begin{figure}
	\includegraphics[width=\textwidth]{metrics_statistics/mamPhastCons46way}
	\caption{Sampling distribution of metric mamPhastCons46way}
\end{figure}
\subsection{Metric values}
\begin{figure}
	\includegraphics[width=\textwidth]{metrics_plot/mamPhastCons46way}
	\caption{Values of metric mamPhastCons46way}
\end{figure}

\clearpage
\section{mamPhyloP46way}
\subsection{Metric sample distribution}
The data points seem to follow a \textbf{Gaussian} distribution with the following parameters:

\begin{align*}
	\mean{X} = 0.7032457913828309 & \qquad \Var{X} = 0.07627203289198752
\end{align*}
\begin{figure}
	\includegraphics[width=\textwidth]{metrics_statistics/mamPhyloP46way}
	\caption{Sampling distribution of metric mamPhyloP46way}
\end{figure}
\subsection{Metric values}
\begin{figure}
	\includegraphics[width=\textwidth]{metrics_plot/mamPhyloP46way}
	\caption{Values of metric mamPhyloP46way}
\end{figure}

\clearpage
\section{numTFBSConserved}
\subsection{Metric sample distribution}
The data points seem to follow a \textbf{exponential} distribution with the following parameters:

\begin{align*}
	\mean{X} = -4.600037873301623e-12 & \qquad \Var{X} = 0.033419421646804975
\end{align*}
\begin{figure}
	\includegraphics[width=\textwidth]{metrics_statistics/numTFBSConserved}
	\caption{Sampling distribution of metric numTFBSConserved}
\end{figure}
\subsection{Metric values}
\begin{figure}
	\includegraphics[width=\textwidth]{metrics_plot/numTFBSConserved}
	\caption{Values of metric numTFBSConserved}
\end{figure}

\clearpage
\section{priPhastCons46way}
\subsection{Metric sample distribution}
The data points seem to follow a \textbf{Gamma} distribution with the following parameters:
\begin{align*}
	\alpha   = 0.2836383862597563    \qquad  \text{loc} = -1.8643137904859329e-31 \qquad \text{scale} = 0.37399746075497264
\end{align*}
\begin{figure}
	\includegraphics[width=\textwidth]{metrics_statistics/priPhastCons46way}
	\caption{Sampling distribution of metric priPhastCons46way}
\end{figure}
\subsection{Metric values}
\begin{figure}
	\includegraphics[width=\textwidth]{metrics_plot/priPhastCons46way}
	\caption{Values of metric priPhastCons46way}
\end{figure}

\clearpage
\section{priPhyloP46way}
\subsection{Metric sample distribution}
The data points seem to follow an \textbf{Beta} distribution with the following parameters:

\begin{align*}
	\alpha   = 2095270.7440875275    & \qquad  \beta = 4.199025269606416        \\
	\text{loc} = -103376.03746996864 & \qquad \text{scale} = 103377.03863437689
\end{align*}
\begin{figure}
	\includegraphics[width=\textwidth]{metrics_statistics/priPhyloP46way}
	\caption{Sampling distribution of metric priPhyloP46way}
\end{figure}
\subsection{Metric values}
\begin{figure}
	\includegraphics[width=\textwidth]{metrics_plot/priPhyloP46way}
	\caption{Values of metric priPhyloP46way}
\end{figure}

\clearpage
\section{rareVar}
\subsection{Metric sample distribution}
The data points seem to follow an \textbf{Beta} distribution with the following parameters:

\begin{align*}
	\alpha   = 14.148202647100376      & \qquad  \beta = 7669045.025220526        \\
	\text{loc} = -0.008523116473417407 & \qquad \text{scale} = 28973.953544984728
\end{align*}
\begin{figure}
	\includegraphics[width=\textwidth]{metrics_statistics/rareVar}
	\caption{Sampling distribution of metric rareVar}
\end{figure}
\subsection{Metric values}
\begin{figure}
	\includegraphics[width=\textwidth]{metrics_plot/rareVar}
	\caption{Values of metric rareVar}
\end{figure}

\clearpage
\section{verPhastCons46way}
\subsection{Metric sample distribution}
The data points seem to follow a \textbf{Gamma} distribution with the following parameters:
\begin{align*}
	\alpha   = 0.4378982063415524    \qquad  \text{loc} = -2.5307968883256733e-31 \qquad \text{scale} = 0.43138079305533483
\end{align*}
\begin{figure}
	\includegraphics[width=\textwidth]{metrics_statistics/verPhastCons46way}
	\caption{Sampling distribution of metric verPhastCons46way}
\end{figure}
\subsection{Metric values}
\begin{figure}
	\includegraphics[width=\textwidth]{metrics_plot/verPhastCons46way}
	\caption{Values of metric verPhastCons46way}
\end{figure}

\clearpage
\section{verPhyloP46way}
\subsection{Metric sample distribution}
The data points seem to follow a \textbf{Gaussian} distribution with the following parameters:

\begin{align*}
	\mean{X} = 0.5723779382558164 & \qquad \Var{X} = 0.0662715947139185
\end{align*}
\begin{figure}
	\includegraphics[width=\textwidth]{metrics_statistics/verPhyloP46way}
	\caption{Sampling distribution of metric verPhyloP46way}
\end{figure}
\subsection{Metric values}
\begin{figure}
	\includegraphics[width=\textwidth]{metrics_plot/verPhyloP46way}
	\caption{Values of metric verPhyloP46way}
\end{figure}

\chapter{Metric distribution summary}
The metrics seem to follow these sample distributions:

\begin{table}
	\begin{tabular}{|l|l|}
		\hline
		\textbf{Metric}     & \textbf{Distribution} \\
		\hline
		CpGobsExp           & Beta                  \\
		\hline
		CpGperCpG           & Beta                  \\
		\hline
		CpGperGC            & Gaussian              \\
		\hline
		DGVCount            & Gamma                 \\
		\hline
		DnaseClusteredHyp   & Gamma                 \\
		\hline
		EncH3K27Ac          & Gamma                 \\
		\hline
		GCContent           & Gaussian              \\
		\hline
		EncH3K4Me3          & Gamma                 \\
		\hline
		ISCApath            & Gamma                 \\
		\hline
		DnaseClusteredScore & Beta                  \\
		\hline
		EncH3K4Me1          & Gamma                 \\
		\hline
		GerpRS              & Gamma                 \\
		\hline
		GerpRSpv            & Gamma                 \\
		\hline
		commonVar           & Exponential Weibull   \\
		\hline
		dbVARCount          & Gamma                 \\
		\hline
		fantom5Perm         & Gamma                 \\
		\hline
		fantom5Robust       & Gamma                 \\
		\hline
		mamPhastCons46way   & Gamma                 \\
		\hline
		priPhastCons46way   & Gamma                 \\
		\hline
		rareVar             & Beta                  \\
		\hline
		verPhastCons46way   & Gamma                 \\
		\hline
		numTFBSConserved    & Exponential           \\
		\hline
		fracRareCommon      & Beta                  \\
		\hline
		priPhyloP46way      & Beta                  \\
		\hline
		verPhyloP46way      & Gaussian              \\
		\hline
		mamPhyloP46way      & Gaussian              \\
		\hline
	\end{tabular}
	\caption{Metrics and their distribution}
\end{table}

\chapter{Scatter plot}
We now proceed to draw a scatter plot trying to identify eventual data correlations.

\section{Scatter plot}
A scatter plot with higher resolution is available in the project repository.
\begin{center}
	\makebox[\textwidth]{
		\includegraphics[width=\paperwidth]{small_scatter_plot}
	}
\end{center}

\section{Identified data correlations}
Data correlations seem to exist between:

\subsection{dbVARCount and DGVCount}
There is a strong correlation between this two metrics:

\begin{figure}
	\begin{subfigure}{0.3\textwidth}
		\includegraphics[width=\textwidth]{correlations/dbVARCount_DGVCount}
		\caption{Correlation of dbVARCount and DGVCount}
	\end{subfigure}
	\begin{subfigure}{0.3\textwidth}
		\includegraphics[width=\textwidth]{correlations/DGVCount_dbVARCount}
		\caption{Correlation of DGVCount and dbVARCount}
	\end{subfigure}
	\caption{The two metrics DGVCount and dbVARCount are strongly correlated}
\end{figure}

The two metrics seem \textbf{highly} correlated, if not the \textbf{same metric}. This means that one of the two could be removed from the dataset, as it does not add any useful information.

\subsection{mamPhyloP46way and verPhyloP46way}
There is a some correlation between this two metrics:

\begin{figure}
	\begin{subfigure}{0.3\textwidth}
		\includegraphics[width=\textwidth]{correlations/mamPhyloP46way_verPhyloP46way}
		\caption{Correlation of mamPhyloP46way and verPhyloP46way}
	\end{subfigure}
	\begin{subfigure}{0.3\textwidth}
		\includegraphics[width=\textwidth]{correlations/verPhyloP46way_mamPhyloP46way}
		\caption{Correlation of verPhyloP46way and mamPhyloP46way}
	\end{subfigure}
	\caption{The two metrics verPhyloP46way and mamPhyloP46way are softly correlated}
\end{figure}

The two metrics seem \textbf{slightly} correlated, but not enough to consider removing one of the two.

\subsection{mamPhastCons46way and verPhastCons46way}
There is a some correlation between this two metrics:

\begin{figure}
	\begin{subfigure}{0.3\textwidth}
		\includegraphics[width=\textwidth]{correlations/mamPhastCons46way_verPhastCons46way}
		\caption{Correlation of mamPhyloP46way and verPhyloP46way}
	\end{subfigure}
	\begin{subfigure}{0.3\textwidth}
		\includegraphics[width=\textwidth]{correlations/verPhastCons46way_mamPhastCons46way}
		\caption{Correlation of mamPhastCons46way and verPhastCons46way}
	\end{subfigure}
	\caption{The two metrics mamPhastCons46way and verPhastCons46way are softly correlated}
\end{figure}

The two metrics seem \textbf{slightly} correlated, but not enough to consider removing one of the two.

\subsection{mamPhastCons46way and verPhastCons46way}
There is a some correlation between this two metrics:

\begin{figure}
	\begin{subfigure}{0.3\textwidth}
		\includegraphics[width=\textwidth]{correlations/mamPhastCons46way_verPhastCons46way}
		\caption{Correlation of mamPhyloP46way and verPhyloP46way}
	\end{subfigure}
	\begin{subfigure}{0.3\textwidth}
		\includegraphics[width=\textwidth]{correlations/verPhastCons46way_mamPhastCons46way}
		\caption{Correlation of mamPhastCons46way and verPhastCons46way}
	\end{subfigure}
	\caption{The two metrics mamPhastCons46way and verPhastCons46way are softly correlated}
\end{figure}

The two metrics seem \textbf{slightly} correlated.

\subsection{verPhastCons46way and priPhastCons46way}
There is a some correlation between this two metrics:

\begin{figure}
	\begin{subfigure}{0.3\textwidth}
		\includegraphics[width=\textwidth]{correlations/priPhastCons46way_verPhastCons46way}
		\caption{Correlation of priPhastCons46way and verPhastCons46way}
	\end{subfigure}
	\begin{subfigure}{0.3\textwidth}
		\includegraphics[width=\textwidth]{correlations/verPhastCons46way_priPhastCons46way}
		\caption{Correlation of verPhastCons46way and priPhastCons46way}
	\end{subfigure}
	\caption{The two metrics verPhastCons46way and priPhastCons46way are softly correlated}
\end{figure}

The two metrics seem \textbf{slightly} correlated.

\subsection{priPhastCons46way and mamPhastCons46way}
There is a some correlation between this two metrics:

\begin{figure}
	\begin{subfigure}{0.3\textwidth}
		\includegraphics[width=\textwidth]{correlations/mamPhastCons46way_priPhastCons46way}
		\caption{Correlation of mamPhastCons46way and priPhastCons46way}
	\end{subfigure}
	\begin{subfigure}{0.3\textwidth}
		\includegraphics[width=\textwidth]{correlations/priPhastCons46way_mamPhastCons46way}
		\caption{Correlation of priPhastCons46way and mamPhastCons46way}
	\end{subfigure}
	\caption{The two metrics priPhastCons46way and mamPhastCons46way are softly correlated}
\end{figure}

The two metrics seem \textbf{slightly} correlated.

\part{Theory}
\chapter{Input modelling}

\section{Input values}
The values used for each metric are the 3 following:

\subsection{Normalized metric}
Clearly one of the important metrics is the metric itself, that will be normalized to allow for input in \(\sqr{-1,1}\) range:

\begin{figure}
	\[
		m' = 2\cdot\frac{\text{metric} - \min\crl{\text{metric values}}}{\max\crl{\text{metric values}}- \min\crl{\text{metric values}}} -1
	\]
	\caption{Input normalization to \(\sqr{-1,1}\) range}
\end{figure}

\subsection{Probability}
Another value we will be using in the input layer of the network is the probability of the metric value, modelled from the estimated sampling distribution of the normalized metric:

If \(M\) is the estimated metric distribution cumulative distribution function (CDF), \(m\) is the value assumed by the metric in the given data point and \(\epsilon \) is a range, we can model the \textbf{probability} in the given value as follows:
\begin{figure}
	\[
		\prob{m' - \epsilon \leq X \leq m' + \epsilon} = M(m' + \epsilon) - M(m' - \epsilon)
	\]
	\caption{Probability}
\end{figure}

\subsection{Deviation from mean}
The third and last input value is the deviation from the normalized value and the mean:

\begin{figure}
	\[
		\text{deviation}(m) = m' - \mean{X}
	\]
	\caption{Deviation from mean}
\end{figure}

\section{Feet}
The input layer is comprised of 25 (number of metrics, excluded the one recognized to be in strong correlation to another) \textit{feet}, meaning tiny networks that are used to limit the initial linear combination of the metric input values to themselves.

Each feet is modelled as a locally connected dense layer, with a window of 3 neurons.

\section{Oversampling of positives}
Since the positive values are just the \(0.036\% \) of the dataset we'll oversample these to weight more these values. Since the variance of positive data points is too high to extrapolate a distribution to generate significant new fuzzy data points, simple duplication will be used.

\section{Undersampling of negatives}
Since the negative values are more than the \(99.96\% \) of the dataset we'll undersample these to weight less these values.

\section{Oversampling and undersampling targets}
Oversampling and undersampling target will be to have a training dataset with \(1\% \) of positives and \(99\% \) of negatives.

\section{Absence of information}
Absence of information about a given metric will be modelled as \textbf{zeros}, meaning all values relative to the given absent metric for that data point will be treated as zero.

\chapter{Output modelling}
The output layer of the neural network is modelled by \textbf{two} neurons, one representing the positive class and one the negative class, with a \textbf{sigmoid} as activation function.

\chapter{Weight initialization}
\section{Weight distribution based on input distribution}
Since input values are not from any particular distribution or hold properties such as \(\mean{X} = 0\) or \(\Var{X} = 1\) (in some metrics mean and variance are far from these values) they do not suggest to use any specific distribution.

\section{Weight distribution based on activation functions and regularization layers}
The codomain values from the activation functions, being SELU for most of the network, tend to hold the properties of \(\mean{X} = 0\) and \(\Var{X} = 1\) (\url{http://arxiv.org/abs/1706.02515}). These values are then regularized to penalize extreme weights that may appear when variance starts with a value significantly away from \(1\).

For these properties weight will be initialized by extracting them from a Gaussian with \(\mean{X} = 0\) and \(\Var{X} = 1\).

\chapter{Locally connected dense layers}
The first two layers will be locally connected dense layers, to exploit the positional information of the input values.

Other than the group of triples, input will be sorted by distribution kind so that the initial interpolations happen mostly with data from the same distribution family.

\begin{figure}
	\includegraphics[width=0.3\textwidth]{locally_connected}
	\caption{Locally connected layer}
\end{figure}

\section{Activation function}
We'll be using a \textbf{leaky relu} for the locally connected dense layers with \(\alpha=0.3\).

\begin{figure}
	\includegraphics[width=0.8\textwidth]{relu}
	\caption{RELU and Leaky RELU}
\end{figure}

\chapter{Dense layers}
For the following hidden layers we will be using dense connected layers, with a piramidal structure (reducing the number of the neurons from 26 to 2).

\begin{figure}
	\includegraphics[width=0.3\textwidth]{dense_connected}
	\caption{Dense connected layer}
\end{figure}

\section{Activation function}
We'll be using again \textbf{leaky relu} and experimenting with \textbf{SELU} for the dense layers:

\begin{figure}
	\[
		\text{selu}(x) = \lambda \begin{cases}
			x                 & x > 0    \\
			\alpha e^x-\alpha & x \leq 0
		\end{cases}
	\]
	\caption{SELU}
\end{figure}

\section{Regularization}
Regularization layers will be alternated to the dense layers to penalize weight extreme growth.

\section{Drop out}
In addition to regularization, also \textbf{drop out} of \(10\% \) of neurons per hidden layer will be applied.

\input{\main/../../general/footer.tex}

\chapter{References}
LatexTools does not compile references at this time.

\end{document}